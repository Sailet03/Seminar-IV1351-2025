\documentclass[a4paper]{scrartcl}
\usepackage[utf8]{inputenc}
\usepackage[english]{babel}
\usepackage{graphicx}
\usepackage{lastpage}
\usepackage{pgf}
\usepackage{wrapfig}
\usepackage{fancyvrb}
\usepackage{fancyhdr}
\usepackage{hyperref}
\pagestyle{fancy}

% Create header and footer
\headheight 27pt
\pagestyle{fancyplain}
\lhead{\footnotesize{Data Storage Paradigms, IV1351}}
\chead{\footnotesize{Project Report}}
\rhead{}
\lfoot{}
\cfoot{\thepage\ (\pageref{LastPage})}
\rfoot{}

\title{Project Report}
\subtitle{Data Storage Paradigms, IV1351}
\author{}
\date{Date}

\begin{document}

\maketitle
\noindent\textbf{Project members:} \\ \hfill
[Elias Tosteberg, eliasto@kth.se] \\ \hfill


\section*{Declaration:}

By submitting this assignment, it is hereby declared that all group members listed above have contributed to the solution. It is also declared that all project members fully understand all parts of the final solution and can explain it upon request.

It is furthermore declared that the solution below is a contribution by the project members only, and specifically that no part of the solution has been copied from any other source (except for lecture slides at the course IV1351), no part of the solution has been provided by someone not listed as a project member above, and no part of the solution has been generated by a system.


\section{Introduction}


I have done the Mandatory Part of Task 1. Logical and Physical Model. Where a model of the database and scripts for generating the database and its data will be created. 

\section{Literature Study}

When preparing for the project information was gathered from the digital Logical and Physical model lecture. There it showed how to create the model.





\section{Method}

When creating the model Astha was used. The first step was to create all the necessary tables. The next Step was to add all the attributes that are not multivariable attributes. Then Primary keys were then decided upon either a unique attribute or a surrogate key. The relations were then added for anything that were not a many-to-many relation. For any many-to-many relations a cross-reference table was added that contained the primary keys of both tables. Any multivariable attributes were then added as a separate table containing the Primary key of the original table and the attribute. After the Astah model was finished Astah was used to generate the creation script of the database. The data script was then created with a combination of online data generation and manual work.

\section{Result}


\begin{figure}[h!]
  \begin{center}
    \includegraphics[scale=0.2]{ER Diagram0.jpg}
    \caption{A sample diagram, included to illustrate caption (this text), numbering and reference in text.}
    \label{fig:diag}
  \end{center}
\end{figure}


The scrips and Astah are stored here: \href{https://github.com/Sailet03/Seminar-IV1351-2025/tree/main/Seminar1}{github.com/Sailet03/Seminar-IV1351-2025}.

\subsection{Number one: Planned hours calculations}





\pagebreak




\section{Discussion}

\textbf{This chapter \textit{analysis} the result presented in the previous section.} 

Evaluate your solution according to the assessment criteria found in the assessment-criteria documents, which are found under the bullet \textit{In the Discussion chapter of your report...}, under each task on the Project page in Canvas. You do not have to cover all specified criteria.

 


\end{document}
