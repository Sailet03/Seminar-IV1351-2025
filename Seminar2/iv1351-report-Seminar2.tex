\documentclass[a4paper]{scrartcl}
\usepackage[utf8]{inputenc}
\usepackage[english]{babel}
\usepackage{graphicx}
\usepackage{lastpage}
\usepackage{pgf}
\usepackage{wrapfig}
\usepackage{fancyvrb}
\usepackage{fancyhdr}
\usepackage{hyperref}

\pagestyle{fancy}

% Create header and footer
\headheight 27pt
\pagestyle{fancyplain}
\lhead{\footnotesize{Data Storage Paradigms, IV1351}}
\chead{\footnotesize{Project Report}}
\rhead{}
\lfoot{}
\cfoot{\thepage\ (\pageref{LastPage})}
\rfoot{}

\title{Project Report}
\subtitle{Data Storage Paradigms, IV1351}
\author{}
\date{Date}

\begin{document}

\maketitle
\noindent\textbf{Project members:} \\ \hfill
[Elias Tosteberg, eliasto@kth.se] \\ \hfill


\section*{Declaration:}

By submitting this assignment, it is hereby declared that all group members listed above have contributed to the solution. It is also declared that all project members fully understand all parts of the final solution and can explain it upon request.

It is furthermore declared that the solution below is a contribution by the project members only, and specifically that no part of the solution has been copied from any other source (except for lecture slides at the course IV1351), no part of the solution has been provided by someone not listed as a project member above, and no part of the solution has been generated by a system.






\section*{Tips for Report Writing}
\textbf{REMOVE THIS SECTION BEFORE SUBMITTING THE REPORT.}\\

\noindent \textit{The target audience has exactly the same skills as the author, except they do not know anything at all about the specific application described in the report.} \\

Consider the following:

\begin{itemize}
  \item \textbf{The report must be \textit{centered around the requirements}. Which are they (Introduction), how did you work to meet them (Method), what is the solution that meets them (Result), and how can you be sure they are met (Discussion). This is the IMRaD method.} The requirements on the Introduction, Method, Result and Discussion chapters are described below under each chapter.

  \item Is spelling and grammar correct? Is spoken language avoided?

  \item Does the report have a good structure with sections, subsections and paragraphs?

  \item Is the text clarified with images and/or other figures, and with links to the code in your Git repository? Remember that all figures (images, tables, graphs, code listings, etc) shall be numbered and have a short explaining text.
\end{itemize}





\section{Introduction}

\textbf{This chapter tells \textit{what} are you going to do.} 

For this seminar the task is to design OLAP queries. These should run on a database Designed for a university.

\section{Literature Study}

This chapter must prove that you collected sufficient knowledge before starting development, instead of just hacking away without knowing how to complete a task. State what you have read and briefly summarize what you have learned.

\section{Method}

\textbf{This chapter tells \textit{how} you solved the task.}

Explain how you worked when solving the tasks and how you evaluated that your solution met the requirements. \textit{Do not explain your solution and do not refer to code}, that belongs to the \textit{Result} chapter. More specific instructions for the content can be found under each task on the Project page in Canvas.

\section{Result}



The queries are stored as Query.sql in \href{https://github.com/Sailet03/Seminar-IV1351-2025/tree/main/Seminar2}{github.com/Sailet03/Seminar-IV1351-2025}.

\subsection{Number one: Planned hours calculations}

The query will calculate a breakdown of all the hours needed for every course instance for a given year. First the query will join the CourseLayout, Instance, PlannedActivities and ActivityType to get all the required information. Since the hours are stored with the only difference being the type we need to do some math to separate the different activity types into separate columns. This is done by using SUM together with WHERE to sum together all activities of the same type into  a value that can be showed in a column. The GROUP BY statement ensures that everything belonging to the same instance is treated together and the WHERE ensures that only the specified year is shown. Here is the output of the query.


\begin{table}[h]
\resizebox{\textwidth}{!}{%
\begin{tabular}{lllllllllllll}
\hline
Course Code & Course Instance ID & HP & Period & \# Students & Lecture Hours & Tutorial Hours & Lab Hours & Seminar Hours & Other Overhead Hours & Admin & Exam & Total Hours \\ \hline
CS101 & 2024-1 & 7 & P1 & 30 & 137 & 34 & 53 & 24 & 9 & 88 & 108 & 453 \\
MA201 & 2024-2 & 5 & P2 & 25 & 169 & 42 & 70 & 30 & 11 & 80 & 100 & 502 \\
PH110 & 2024-3 & 10 & P3 & 40 & 137 & 34 & 55 & 24 & 9 & 100 & 122 & 481 \\ \hline
\end{tabular}%
}
\end{table}

\subsection{Number two: Calculate actual allocated hours for a course}

This query will break down how much time is allocated to per employee for a specific course instance. This query works mostly the same since most of the columns are the same. Since it required more information than number one it also had to JOIN the EmployeesPlannedActivities, Employees, Titles and Person. The WHERE statement ensures that the correct course instance is selected. The GROUP By statement is ensuring that the employees are grouped together.

\begin{table}[h]
\resizebox{\textwidth}{!}{%
\begin{tabular}{lllllllllllll}
\hline
Course Code & Course Instance ID & HP & Teacher's Name & Designation & Lecture Hours & Tutorial Hours & Lab Hours & Seminar Hours & Other Overhead Hours & Admin & Exam & Total Hours \\ \hline
CS101 & 2024-1 & 7 & Alice Johnson & Assistant Professor & 0 & 0 & 24 & 0 & 5 & 0 & 0 & 29 \\
CS101 & 2024-1 & 7 & Bob Smith & Associate Professor & 0 & 0 & 29 & 0 & 4 & 0 & 0 & 33 \\
CS101 & 2024-1 & 7 & Frank Miller & Lecturer & 72 & 0 & 0 & 11 & 0 & 0 & 54 & 137 \\
CS101 & 2024-1 & 7 & Grace Lee & Senior Lecturer & 65 & 0 & 0 & 13 & 0 & 0 & 54 & 132 \\
CS101 & 2024-1 & 7 & Hannah Kim & Lecturer & 0 & 16 & 0 & 0 & 0 & 44 & 0 & 60 \\
CS101 & 2024-1 & 7 & Ian Clark & Lab Assistant & 0 & 18 & 0 & 0 & 0 & 44 & 0 & 62 \\ \hline
\end{tabular}%
}
\end{table}

\subsection{Number three: Calculate the total allocated hours for a teacher, only for the current years’ course instances}

 This query is also similar to the first 2. The query gets the hours that are assigned to a specific employee for all course instances in a certain year. Here The WHERE statement filters on year and name of an employee. The rows are grouped by the teacher name and year.

\begin{table}[h]
\resizebox{\textwidth}{!}{%
\begin{tabular}{llllllllllll}
\hline
Course Code & Course Instance ID & HP & Teacher's Name & Lecture Hours & Tutorial Hours & Lab Hours & Seminar Hours & Other Overhead Hours & Admin & Exam & Total Hours \\ \hline
CS101 & | 2024-1 & 7 & Frank Miller & 72 & 0 & 0 & 11 & 0 & 0 & 54 & 137 \\
MA201 & | 2024-2 & 5 & Frank Miller & 0 & 20 & 0 & 0 & 0 & 40 & 0 & 60 \\
PH110 & | 2024-3 & 10 & Frank Miller & 0 & 0 & 29 & 0 & 4 & 0 & 0 & 33 \\ \hline
\end{tabular}%
}
\end{table}

\subsection{Number four: List employee ids and names of all teachers who are allocated in more than a specific number of course instances during the current period}

The final Query shows all the employees that have scheduled activities belonging to more than a certain number of instances in a specific period. The query works by joining Employees, Person, EmployeesPlannedActivities, PlannedActivities and Instance together. The Query then uses COUNT DISTINCT to count the number of instances in the current period. The HAVING filter is then used to ensure only the teachers with more than the lower amount is shown.




\begin{table}[h]
\centering
\begin{tabular}{llll}
\hline
Employment ID & Teacher's Name & Period & No of Courses \\ \hline
3 & Alice Johnson & P1 & 1 \\
4 & Bob Smith & P1 & 1 \\
1 & Frank Miller & P1 & 1 \\
2 & Grace Lee & P1 & 1 \\
5 & Hannah Kim & P1 & 1 \\
6 & Ian Clark & P1 & 1 \\ \hline
\end{tabular}
\end{table}



\begin{figure}[h!]
  \begin{center}
    \includegraphics[scale=0.15]{ER Diagram0.jpg}
    \caption{Database schema}
    \label{fig:diag}
  \end{center}
\end{figure}

\pagebreak

\section{Discussion}

\textbf{This chapter \textit{analysis} the result presented in the previous section.} 

Evaluate your solution according to the assessment criteria found in the assessment-criteria documents, which are found under the bullet \textit{In the Discussion chapter of your report...}, under each task on the Project page in Canvas. You do not have to cover all specified criteria.



\end{document}
